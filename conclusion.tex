\chapter*{General conclusion}
\addcontentsline{toc}{chapter}{General conclusion}
\label{chap.general-conclusion}

The increasing global demand for wheat, coupled with the persistent threat of plant diseases and insect pests, underscores the urgent need for innovative and efficient monitoring systems in agriculture. Traditional disease detection methods, while useful, are often limited in scale, accuracy, and timeliness. This has prompted the adoption of smart agriculture technologies, particularly the integration of remote sensing with machine learning (ML) and deep learning (DL) techniques.

Throughout this report, we have examined how the fusion of aerial imagery acquired through satellites, UAVs, and multispectral sensors with advanced AI models enables early detection, classification, and monitoring of various wheat diseases. We have explored the fundamentals of DL and ML, including their roles in core computer vision tasks, and reviewed different imaging technologies used in precision agriculture.

Moreover, we discussed the challenges inherent in this integration, such as data heterogeneity, high computational demands, and the need for large annotated datasets. Despite these limitations, the results from existing studies demonstrate that remote sensing combined with AI holds immense potential for automated, scalable, and real-time disease detection systems.

In conclusion, the integration of remote sensing and intelligent algorithms is not only transforming the way we monitor crop health but also paving the way for sustainable, data-driven agricultural practices. Future work should focus on improving data fusion strategies, model generalization, and deploying lightweight solutions for field-level implementation bringing us closer to the vision of fully autonomous smart farming systems.