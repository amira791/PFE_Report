% \ \vfill{}

\thispagestyle{plain}
\pagenumbering{roman}
\setcounter{page}{2}
\phantomsection\addcontentsline{toc}{chapter}{\abstractname}

\noindent
{\Large\bfseries Abstract}

\noindent
Wheat is a vital crop that significantly contributes to global food security, yet its production is increasingly threatened by various diseases. Traditional detection and management approaches often fall short due to their reliance on manual inspection and delayed intervention. In response, the integration of smart agricultural technologies, particularly remote sensing and artificial intelligence, has emerged as a powerful solution for early and accurate disease detection. 

This report explores the application of remote sensing data in combination with machine learning and deep learning techniques to address the challenges of wheat disease monitoring. It presents a comprehensive overview of imaging technologies such as RGB, thermal, multispectral, and hyperspectral cameras, as well as data collection platforms including satellites, aircraft, and UAVs. 

The study examines the full workflow from image acquisition and preprocessing to feature extraction and model-based classification using a variety of machine learning and deep learning models. It also explains how combining data from different sources can improve detection accuracy, and highlights future opportunities to use these technologies for more efficient and sustainable farming.    

\rule{\linewidth}{1pt}
\textbf{Keywords --- } Remote sensing, Precision agriculture, Machine learning (ML), Deep learning (DL), Vegetation indices (NDVI, NDRE).\\
\rule{\linewidth}{1pt}

\newpage

\begin{otherlanguage}{french}
    \thispagestyle{plain}
    \pagenumbering{roman}
    \setcounter{page}{3}
    \phantomsection\addcontentsline{toc}{chapter}{\abstractname}

    \noindent
    {\Large\bfseries Résumé}

    \noindent
    Le blé est une culture essentielle qui contribue de manière significative à la sécurité alimentaire mondiale, mais sa production est de plus en plus menacée par diverses maladies. Les méthodes traditionnelles de détection et de gestion, reposant souvent sur des inspections manuelles et des interventions tardives, montrent leurs limites. Face à ces défis, l’intégration des technologies agricoles intelligentes, notamment la télédétection et l’intelligence artificielle, s’impose comme une solution efficace pour une détection précoce et précise des maladies.

    Ce rapport explore l’utilisation des données de télédétection combinées à des techniques d’apprentissage automatique et d’apprentissage profond pour surveiller les maladies du blé. Il propose un aperçu complet des technologies d’imagerie telles que les caméras RGB, thermiques, multispectrales et hyperspectrales, ainsi que des plateformes de collecte de données, incluant les satellites, les avions et les drones (UAVs).

    L’étude examine l’ensemble du processus, de l’acquisition et du prétraitement des images à l’extraction des caractéristiques et à la classification basée sur différents modèles d’apprentissage. Elle met également en évidence comment la combinaison de données provenant de différentes sources peut améliorer la précision de la détection, tout en soulignant les perspectives futures pour une agriculture plus efficace et durable.

    \vspace{1cm}
    \noindent\rule{\linewidth}{1pt}
    \noindent
    \textbf{Mots clés --- } Télédétection, Agriculture de précision, Apprentissage automatique (ML), Apprentissage profond (DL), Indices de végétation (NDVI, NDRE).\\
    \noindent\rule{\linewidth}{1pt}
\end{otherlanguage}

\newpage

% \renewcommand{\abstractname}{\RL{مـلـخـص}}

% \afterpage{
%     \thispagestyle{plain}
%     \pagenumbering{roman}
%     \setcounter{page}{4}  % Keep if required
%     \phantomsection
%     \addcontentsline{toc}{chapter}{\texorpdfstring{\RL{مـلـخـص}}{ملخص}}

%     \begin{flushright}
%     {\Large\bfseries \textRL{مـلـخـص}}\\[1em]

%     \noindent
%     \begin{RLtext}
%     القمح هو محصول حيوي يسهم بشكل كبير في الأمن الغذائي العالمي، ومع ذلك فإن إنتاجه مهدد بشكل متزايد من قبل العديد من الأمراض. غالبًا ما تكون أساليب الكشف والإدارة التقليدية غير كافية نظرًا لاعتمادها على الفحص اليدوي والتدخل المتأخر. استجابةً لذلك، ظهرت تقنيات الزراعة الذكية، وخاصة الاستشعار عن بُعد والذكاء الاصطناعي، كحل قوي للكشف المبكر والدقيق عن الأمراض.

%     \medskip

%     يستعرض هذا التقرير تطبيق بيانات الاستشعار عن بُعد بالاشتراك مع تقنيات التعلم الآلي والتعلم العميق للتصدي لتحديات مراقبة أمراض القمح. يقدم التقرير لمحة شاملة عن تقنيات التصوير مثل كاميرات التصوير ثلاثي الألوان ، والتصوير الحراري، والمجموعة الطيفية المتعددة، والكاميرات الطيفية الفائقة، بالإضافة إلى منصات جمع البيانات مثل الأقمار الصناعية والطائرات والطائرات بدون طيار.

%     \medskip

%     كما يفحص التقرير سير العمل الكامل من الحصول على الصور والمعالجة الأولية لها، إلى استخراج الميزات وتصنيف البيانات باستخدام مجموعة من نماذج التعلم الآلي والتعلم العميق. كما يشرح كيف يمكن لتحليل البيانات من مصادر متعددة تحسين دقة الكشف، ويبرز الفرص المستقبلية لاستخدام هذه التقنيات من أجل الزراعة الأكثر كفاءة واستدامة.
%     \end{RLtext}

%     \vspace{1cm}

%     % Instead of \rule{\linewidth}{1pt}, use a line that will work in the RTL environment
%     \noindent\rule[0pt]{0.5\linewidth}{1pt} % Adjust the length as needed

%     \begin{RLtext}
%     \textbf{الـكـلـمـات الـمـفـتـاحـيـة:} الاستشعار عن بعد، الزراعة الدقيقة، التعلم الآلي، التعلم العميق، مؤشرات الغطاء النباتي.
%     \end{RLtext}

%     \noindent\rule[0pt]{0.5\linewidth}{1pt} % Adjust the length again if needed
%     \end{flushright}

%     \restoregeometry
% }


\renewcommand{\abstractname}{\textRL{مـلـخـص}}

\afterpage{
    \pagenumbering{roman}
    \setcounter{page}{4}
    \phantomsection
    \addcontentsline{toc}{section}{\textRL{مـلـخـص}}

    \begin{abstract}
        \thispagestyle{plain}
        \begin{RLtext}
 القمح هو محصول حيوي يسهم بشكل كبير في الأمن الغذائي العالمي، ومع ذلك فإن إنتاجه مهدد بشكل متزايد من قبل العديد من الأمراض. غالبًا ما تكون أساليب الكشف والإدارة التقليدية غير كافية نظرًا لاعتمادها على الفحص اليدوي والتدخل المتأخر. استجابةً لذلك، ظهرت تقنيات الزراعة الذكية، وخاصة الاستشعار عن بُعد والذكاء الاصطناعي، كحل قوي للكشف المبكر والدقيق عن الأمراض.

    \medskip

    يستعرض هذا التقرير تطبيق بيانات الاستشعار عن بُعد بالاشتراك مع تقنيات التعلم الآلي والتعلم العميق للتصدي لتحديات مراقبة أمراض القمح. يقدم التقرير لمحة شاملة عن تقنيات التصوير مثل كاميرات التصوير ثلاثي الألوان ، والتصوير الحراري، والمجموعة الطيفية المتعددة، والكاميرات الطيفية الفائقة، بالإضافة إلى منصات جمع البيانات مثل الأقمار الصناعية والطائرات والطائرات بدون طيار.

    \medskip

    كما يفحص التقرير سير العمل الكامل من الحصول على الصور والمعالجة الأولية لها، إلى استخراج الميزات وتصنيف البيانات باستخدام مجموعة من نماذج التعلم الآلي والتعلم العميق. كما يشرح كيف يمكن لتحليل البيانات من مصادر متعددة تحسين دقة الكشف، ويبرز الفرص المستقبلية لاستخدام هذه التقنيات من أجل الزراعة الأكثر كفاءة واستدامة.
  

 \end{RLtext}

        \hspace*{0mm}\rule{\linewidth}{1pt}
        \begin{RLtext}
        \textbf{الـكـلـمـات الـمـفـتـاحـيـة ـــ }الزراعة الذكية، التعلم العميق، الرؤية الحاسوبية، أمراض القمح، كشف الآفات الحشرية.
        \end{RLtext}
        \rule{\linewidth}{1pt}
    \end{abstract}
    \restoregeometry
}
