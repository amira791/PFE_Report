\chapter*{General introduction}
\addcontentsline{toc}{chapter}{General introduction}
\label{chap.intro}

Wheat, a staple food crop and a cornerstone of global food security, faces persistent threats from a wide array of diseases and pests, which can drastically diminish both yield and quality. Traditionally, farmers have relied on manual inspection and chemical treatments to manage crop diseases, but these methods are often time-consuming, imprecise, and unsustainable in the long term.

In recent years, the rise of smart agriculture has introduced innovative approaches to crop monitoring and disease management. In particular, the integration of remote sensing technologies with machine learning (ML) and deep learning (DL) algorithms has enabled a paradigm shift in how agricultural data is collected, processed, and utilized.

Remote sensing platforms including satellites, drones (UAVs), and ground-based sensors allow the non-invasive acquisition of multispectral, hyperspectral, and thermal imagery, which can reveal subtle signs of plant stress and disease that are invisible to the naked eye. When combined with ML and DL models, this data can be automatically analyzed to classify disease types, detect early infections, and support real-time decision-making at scale.

This report investigates the synergy between remote sensing and artificial intelligence in the context of wheat disease detection. It provides:


\begin{itemize}
    \item An overview of common wheat diseases and their agricultural impact.
    \item A detailed examination of Deep Learning methods and their application in computer vision tasks such as image classification, object detection, and segmentation.
    \item A technical survey of remote sensing tools and platforms used for agricultural monitoring.
    \item A comparative analysis of existing approaches and their application by discussing workflows for integrating imagery with AI models, feature extraction methods, and data fusion strategies.
  \end{itemize}
  
By exploring these technologies and their integration, this work highlights the potential of intelligent systems to improve disease management, reduce crop losses, and contribute to more resilient and sustainable agriculture.